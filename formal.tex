\documentclass[12pt]{article}
\usepackage{tikz}
\usepackage{pgfplots}
 % Useful for Physics related math
\usepackage{physics}
\usepackage{amsmath}
\usepackage{amssymb} 
\usepackage{amsthm}
        % packages that allow mathematical formatting
\usepackage{setspace}
\setlength\parindent{0pt}
        % set space and indent and indent
\usepackage{hyperref}
\hypersetup{
            colorlinks,
                linkcolor={red!50!black},
                    citecolor={blue!50!black},
                        urlcolor={blue!80!black}
                }
\usepackage[a4paper,width=150mm,top=25mm,bottom=25mm]{geometry}

% New commands
\let\conju\overline
\newcommand\numberthis{\addtocounter{equation}{1}\tag{\theequation}}
\newcount\colveccount
\newcommand*\colvec[1]{
        \global\colveccount#1
        \begin{pmatrix}
                \colvecnext
        }
        \def\colvecnext#1{
                #1
                \global\advance\colveccount-1
                \ifnum\colveccount>0
                \\
                \expandafter\colvecnext
                \else
        \end{pmatrix}
        \fi
}
        % Columnvectors
\newtheorem{mydef}{Definition}

\pgfplotsset{compat=1.11}
\linespread{1.5}
\newcommand{\sthuga}{Stag hunt game}
\newcommand{\natnumb}{\mathbb{N}}
\newcommand{\realnumb}{\mathbb{R}}
\newcommand{\triplet}{$\Gamma$}
\newcommand{\player}{I}
\newcommand{\strat}{S}
\newcommand{\stu}{s}
\newcommand{\pay}{F}
\newcommand{\sprob}{\hat{s}}
\newcommand{\svec}{\vec{\sprob}}
\renewcommand{\vec}{\mathbf}
\newcommand{\mixpay}{\hat{\pay}}
\newcommand{\mixstratspace}{\Omega}
\newcommand{\payma}{\mathbb{A}}
\begin{document}
\section{The Stag hunt game}
Game theory tries to describe the behaviour of agents in strategic interactions.
By strategic interaction a game theorist means that the so called agents
can choose between actions, effecting not only their personal outcome, but 
also their opponent's. Such a situation is called a \textit{game}. 
A \textit{game} is represented in \textit{normal form} if the agents choose
their strategies simultaneously, without knowing what the other agents chose. 
More formaly a game is usually represented as a triplet \triplet$=(\player,
\strat,\pay)$, where $\player$ is the set of players playing the game,
$\strat$ the strategy space of the game and $\pay$ the payoff function
displaying the outcome of the game. 
The previoulsy introduced \sthuga is a game in \textit{normal form} as both 
players, the two hunters, choose simulatenously whether to hunt the stag or
the hare. Formally, the set of players $\player \in  {1,2}$ are the two hunters.
Each hunter $i \in \player$ can choose from a pure strategy in their pure
strategy space $\strat_i = \{1,2\}$, where strategy 1 is the strategy hunting
stag and strategy 2 represents hunting hare. The \textit{pure-strategy space}
of the game is defined as $\strat = \strat_1 \times \strat_2$, the cartesian
product of the individual pure strategy spaces. A vector $\stu = (\stu_1,\stu_2)  \in  \strat$
is called a \textit{pure-strategy profile} indicating which pure strategies 
$s_i$ with $i \in \player$ the players chose. The last element for a formal 
description is the map determining the outcome of the game, or in the analogy
of the \sthuga, with what loot each hunter comes home. Therefore, let 
the individual payoff function be $\pay_i: \strat \rightarrow \realnumb$ for
a player $i \in \player$. This maps to every pure-strategy profile $\stu \in S$ a
payoff. In terms of the \sthuga, this defines for every strategy a hunter can
choose a payoff, the amount of loot he gets, depending on the strategy 
of the other hunter. Extending this to a pure-strategy payoff function 
$\pay: \strat \rightarrow \realnumb^n$ one can describe the payoff to all
players to a particular strategy profile $\stu \in \strat$ with a vector
$\pay(\stu) = \left(\pay_1,\pay_2\right)$. 
Typically the formulation of a normal form game is extended by the abillity
for each player to randomize the choice of their strategies. A game is then 
called a game with \textit{mixed strategies}. One can either
interpret this in the for a lot of games rather unconvincing interpretation
that a player truly flips a coin or some other randomizing process to choose
between their strategies. 
\begin{enumerate}
        \item Mixed strategies Interpretation
       \item Sources
        \item Later on, I will introduce the concept of a population game in
                which mixed strategies can be interpreted quite differently.
\end{enumerate} 
Therefore, mixed strategies can be formalized as that every player $i \in 
\player$ assignes (positive) probabilities $\sprob_{ij}$ to each of his pure
strategies $\stu_j \in \strat_i$, with $j \in \{1,2\}$. The probabilities 
$\sprob_{ij}$ are subject to the normalization conditions
\begin{alignat*}{2}
        \sum_{j=1}^{2} \sprob_{ij}= 1 \qquad \forall i \in I, \sprob_{ij} \geq 0.
\end{alignat*}
With this addition, one defines player $i$'s mixed-strategy space as 
\begin{align*}
        \Delta_i = \{\svec_i = \left(\sprob_{i1}, \sprob_{i2}\right)| \sprob_{i1} + 
        \sprob_{i2} = 1, \sprob_{i1} \geq 0, \sprob_{i2} \geq 0\}
\end{align*}
\begin{enumerate}
        \item   Geometrically, this can be interpreted as a simplex
\end{enumerate}
The vector $\svec_i$ is therefore called a \textit{mixed-strategy profile} of player
$i \in I$. For example a hunter could flip a (perfect) coin determining if 
he chooses to hunt stag or hare. This strategy would be expressed as
$\svec_i = \left( \sprob_{i1},\sprob_{i2}\right)= \left(0.5,0.5\right)$, 
assigning a 50\% chance to strategy 1 and 50\% to strategy 2. 
A pure strategy can also be interpreted as a mixed strategy. In that case a 
player $i$ chooses to assign a probability of $\sprob_{ij}=1$ to strategy
$j \in \{1,2\}$, while then $\sprob_{ik}=0, \ \forall k \neq j$ i.e. player $i$ 
just uses strategy $j$. Particularly, $\sprob_i = \left(1,0\right)$ and 
$\sprob_i = \left(0,1\right)$ describe the use of the pure strategy 1 and 2 of 
player i in the mixed-strategy space, respectively. 
As for the pure strategy spaces one typically defines the \textit{mixed-strategy}
space of the game as $\Omega = \Delta_1 \times \Delta_2$. One writes 
$\sprob = \left(\sprob_1,\sprob_2\right)$ for the mixed strategy profile of 
the game. 
Additionally, one defines the \textit{mixed payoff function} of the game,
the payoffs to the players using mixed strategies as 
$\mixpay: (\mixpay_1,\mixpay_2): \mixstratspace \rightarrow \realnumb^2$, 
where the individual payoff to player $i \in I$ choosing the mixed-strategy 
profile $\svec_i$ against $\svec_k$,  $k \in I, k \neq i$, $\mixpay_i$ is 
defined as
\begin{align}
        \mixpay_i(\svec_1,\svec_2) = \sum_{j_{1}=1}^{2} \sum_{j_{2}=1}^{2} 
        \pay_{i}(\stu_{i{j_1}},\stu_{i{j_2}}) \cdot \sprob_{1j_1} \cdot 
        \sprob_{2j_1}
\end{align}
In a two player game it is useful to define payoff matrices $\payma$ for player
1 and $\mathbb{B}$ for player 2. The entries are defined as $\payma_{jk} = 
\pay_1(s_{1j},s_{2j})$ and $\mathbb{B}_{jk} = \pay_2(s_{1j},s_{2j})$. 
The game is called \itemit{symmetric} if it holds that $\mathbb{B} = \payma^T$,
indicating that both players of the game choose from the same set of strategies
and face the same payoffs. In the \sthuga\ this is clearly the case as both
hunters choose from two pure strategies and we assume that both receive the
same payoff from selling their loot. 


\end{document}

