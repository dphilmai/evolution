\documentclass[10pt]{article}
\usepackage{tikz}
\usepackage{pgfplots}
 % Useful for Physics related math
\usepackage{physics}
\usepackage{amsmath}
\usepackage{amssymb} 
\usepackage{amsthm}
        % packages that allow mathematical formatting
\usepackage{setspace}
\setlength\parindent{0pt}
        % set space and indent and indent
\usepackage{hyperref}
\hypersetup{
            colorlinks,
                linkcolor={red!50!black},
                    citecolor={blue!50!black},
                        urlcolor={blue!80!black}
                }
\usepackage[a4paper,width=150mm,top=25mm,bottom=25mm]{geometry}

% New commands
\let\conju\overline
\newcommand\numberthis{\addtocounter{equation}{1}\tag{\theequation}}
\newcount\colveccount
\newcommand*\colvec[1]{
        \global\colveccount#1
        \begin{pmatrix}
                \colvecnext
        }
        \def\colvecnext#1{
                #1
                \global\advance\colveccount-1
                \ifnum\colveccount>0
                \\
                \expandafter\colvecnext
                \else
        \end{pmatrix}
        \fi
}
        % Columnvectors
\newtheorem{mydef}{Definition}

\pgfplotsset{compat=1.11}
\linespread{1.5}
\newcommand{\natnumb}{\mathbb{N}}
\newcommand{\realnumb}{\mathbb{R}}
\begin{document}
{\noindent\Huge\bf  \\[0.5\baselineskip] {\fontfamily{cmr}\selectfont  Übungsblatt 11 - Plots}         }\\[2\baselineskip] % Title
{ {\bf \fontfamily{cmr}\selectfont Mathematik für Physiker 1}\\ {\textit{\fontfamily{cmr}\selectfont \today   }}}~~~~~~~~~~~~~~~~~~~~~~~~~~~~~~~~~~~~~~~~~~~~~~~~~~~~~~~~~~~~~~~~~~~~~~~~~~~~~\
{\textsc{Daniel Mairhofer 5610026}}% Author name
\\[1.4\baselineskip] 
    
\section{Stag Hunt game}
A two player, two strategy normalform game in game theory is usually defined 
as a triplet $\Gamma := (I,S,F)$, consisting of the set of players 
$I = \{1,2\}$, the space of pure stragies of the game $S= S^1 \times S^2$,
where $S^i = \left\{\left(s_1^i, s_2^i\right)\right\}$ is the pure stragey 
space of player $ i \in I$. Whereas $S^i$ describes all possible pure stragies
an individual $i$ can choose, $S$ describes all possible states of the game
in pure strategies.
\\
\\

The first element of a formal description of the strategic interactions of 
agents\footnote{Defining the agent as individual, group, e.t.c.} game theory
tries to describe is the set of players $I = \{1,2,...,n\}$, consisting of 
$ n \in \natnumb$ players. As we will focus on two player games, we set $n=2$.
Secondly, each agent playing the game needs a bunch of strategies\footnote{
In gametheory a strategy may be a performed action, an attidude with which a 
certain action is performed or even a phenotyp in biological contexts} he can
choose of. So we define for each agent 
$i \in I$ the pure strategy space 
$S^i = \{(s_1,s_2,...,s_{m_i})\}$, where $s_{j}$ represents pure strategy $j \in 
\{1,2,...,m_i\}$ and $m_i \in \natnumb$ the total
number of pure strategies of player i. In the introduced Stag Hunt game the 
total number of pure strategies is $m_i = 2$ for each player. 
The players decide between shooting the stag or the hare. 
We define hunting the stag as strategy $s_1$ for each player and hunting the 
hare as strategy $s_2$. Therefore, the (pure) strategy space of each player
in this game is symmetric as each hunter can choose from the same set of 
strategies. For later use we define the pure strategy space $S = S^1 \times 
S^2$. 

The last element of an usual game is the payoff function. A map $F:S \rightarrow
\realnumb^2$ is called the payoff function of the game which assigns to each element of
the set of pure strategies an element of $\realnumb^2$, the payoff to the players. 
In a normalform game this payoff function can be represented with a matrix for
each player $F^i \in \realnumb^{2 \times 2}$. As the stag hunt game is a
symmetric game which means that the roles of the players are identical, their
payoff matrices fulfill the relation $F^2 = (F^1)^T$. 

\end{document}
