%%%%%%%%%%%%%
%
%
% Introduction
%
%
%%%%%%%%%%%%%
The beginning of game theory dates back to Johann von Neumann in 1928,
who developed a mathematical framework
for modelling situations with strategic interactions of rational agents 
\parencite{v._neumann_zur_1928}.
In a situation of strategic interaction, the outcome for an agent does not
only depend on his choice, but also on the choice of the other agents.
In economics, such situations often present themselves as a social dilemma,
a situation where a group of agents is prevented to achieve the best
outcome for the group, because of strategic considerations of the individual
agent.

Asking a social scientist about \textit{the} example of a social dilemma, 
one usually gets to hear the story about two prisoners, arrested and 
seperately interrogated by an authority. Lacking the amount of evidence to sue 
the criminals for serious crimes, the authority offers them a deal 
simultaneously.
A confession would grant a prisoner amnesty and hence, he could escape the 
punishment for the serious crimes. However, if both chose to confess, 
they receive the
full sharpness of the law. Not admitting the crimes would leave the authority
with too little evidence, resulting only in a sentence for minor crimes.
This story, developed by Albert Tucker in 1950 and suitably named Prisoner's
Dilemma (PD) is usually represented in a table, as shown in figure \ref{fig:pd}, with
the options for each prisoners denoted by C, remaining silent, and D, admitting
their crimes. 
\begin{figure}[h]
        \centering
        \def\gamestretch{2.1}
        \begin{game}{2}{2}[Player 1][Player 2] & $C$ & $D$
                \\ $C$ &$\stackedpayoffs{-1}{-1}$ &$\stackedpayoffs{-4}{\phantom{-}0}$
        \\ $D$ &$\stackedpayoffs{\phantom{-}0}{-4}$ &$\stackedpayoffs{-3}{-3}$ \end{game}
\caption[Prisoner's Dilemma]{A parametrized representation of the prisoners dilemma}
\label{fig:pd}
\end{figure}
The total amount of years for the group is lowest if both do not confess, 
but each individual prisoner has the 
incentive to take the deal of the authority as this leaves him, independently
of what his fellow prisoner does, with a shorter time in prison. The dilemma 
clearly consists of the conflict between individual interest and 
the benefit of the group \parencite{skyrms_stag_2004}. 

Skyrms argues that there is ``another story that became a game'' 
which describes a different but underrepresented social dilemma
\parencite[1]{skyrms_stag_2004}. 
In \textit{A Discourse on Inequality}, 
Rousseau outlines the situation of hunters, heading out to hunt 
stag. However, if a hare runs past a hunter, he might consider to
shoot the hare, scaring of all stags in the forest. The total reward for
the group is surely lower, but it is rational for the single hunter if he
expects the others to act the same, given the opportunity. In contrast
to PD, it is not the hunters individual interest that is competing with the
benefit of the group, but the uncertainty about what the other hunters choose
to do. In fact, none of the hunters would mind his choice
not to shoot a hare when returning with a stag, as his personal benefit is 
greater with a stag as bounty. 
\begin{figure}[h]
\begin{subfigure}{0.5\textwidth}
\begin{center}
        \def\gamestretch{2.1}
        \begin{game}{2}{2}[Player 1][Player 2] & $S$ & $H$
                \\ $S$ &$\stackedpayoffs{5}{5}$ &$\stackedpayoffs{0}{4}$
        \\ $H$ &$\stackedpayoffs{4}{0}$ &$\stackedpayoffs{3}{3}$ \end{game}
\end{center}
\caption{Numerical example}
\label{fig:numericalsh}
\end{subfigure}
\begin{subfigure}{0.5\textwidth}
\begin{center}
        \def\gamestretch{2.1}
        \begin{game}{2}{2}[Player 1][Player 2] & $S$ & $H$
                \\ $S$ &$\stackedpayoffs{a}{a}$ &$\stackedpayoffs{b}{c}$
        \\ $H$ &$\stackedpayoffs{c}{b}$ &$\stackedpayoffs{d}{d}$ \end{game}
\end{center}
\caption{Parametrized: $a > c \geq d > b$}
\label{fig:parash}
\end{subfigure}
\caption[Stag hunt game]{The stag hunt game: Numerical 
example and parametrized}
\label{fig:sh}
\end{figure}
Figure \ref{fig:sh} represents this dilemma 
for two hunters with a numerical example and parametrized. 
Considering the numerical example, when both coordinate on stag hunting, 
denoted by S, they receive $5$. 
If one of them decides to shoot the hare, he gets $4$, whereas
the other hunter is left with nothing. Coordination on hare
hunting yields both with $3$. Clearly, hunting hare is safer as it yields 
a hunter at least with $3$, whereas stag hunting may leave a hunter with 
nothing. In the representation with parameters in figure
\ref{fig:parash}, these 
are usually assumed to satisfy the condition $a > c \geq d >b$. The 
relevance of this restriction will be discussed throughout the text.
Game theory calls this kind of game a coordination game.

In both outcomes, collective stag hunting and collective hare hunting,
each individual hunter has no incentive to change his action.
Hence, both outcomes are what game theory calls a Nash equilibrium, originally
formulated by John Nash in 1950 \parencite{nash_equilibrium_1950}. In a 
Nash equilibrium all players of a game choose a best reply against
the strategies of the other players, so that none of them has an incentive
to deviate unilateral from his decision. Although this is the mainly used
solution concept, it does not give a definite answer in the stag hunt game.

One might argue that both hunters can speak to each other beforehand and
assure each other that hare hunting is more attractive to both them. 
However, \textcite{camerer_behavioral_2003}
argues that empirical observation and theory suggest that the problem is 
not solved that easily.
Furthermore, in a wide range of economic applications interaction of agents
happen repeatedly. Analyzing repeated prisoner dilemmas, it turns 
out that the resulting game can actually be interpreted as a stag-hunt game with
two Nash equilibria \parencite{skyrms_stag_2004}.

The connection between these two social dilemmas advocates that
the selection of equilibria should be investigated more deeply. 
An interesting approach to the equilibrium selection problem comes from
biology. Evolutionary game theory was pioneered by Maynard J. Smith and George
R. Price with their work on the conflict of animals in populations 
\parencite{smith_lhe_1973}. Started as a refinement of the Nash equilibrium,
the concept of an evolutionary stable strategy, various other authors, for 
example \textcite{taylor_evolutionary_1978}, \textcite{hofbauer_note_1979} and
\textcite{zeeman_dynamics_1981}, developed a ``time dependent dynamical
extension of game theory'' \parencite[55]{hanauske_evolutionare_2011}.
In contrast to traditional game theory, the evolutionary approach  models
agents in a large population, interacting repeatedly in a strategic environment. 
Furthermore, the agents are not assumed to be rational, but follow a specific 
rule, updating their strategy. 
As outlined in the rest of this thesis, the
evolutionary approach offers an answer to the coordination
problem in the stag hunt game.

The thesis is organized as follows. Section 2 introduces the 
framework of traditional game theory for the stag hunt game and discusses
the traditional approach to the equilibrium selection problem. Section 3
introduces evolutionary game theory and presents the replicator dynamic. 
Later on, it will be discussed how the evolutionary approach selects
between multiple equilibria. Section 4 demonstrates the effect of a
network externality in the underlying framework. Looking for evidence on
how real people play the stag hunt game, section 5 considers experimental
literature on the topic. Section 6 concludes and shows the application of
evolutionary game theory and the stag hunt game in economics.

