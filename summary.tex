%%%%%%% 
%%
%% Summary and Outlook
%%
%%%%%%%

The origin of the social dilemma described in this text does not arise
from a conflict between individual interest and the interest of the group.
It is a dilemma because the individuals involved are not certain about
what their counterpart does. While one outcome\todo{payoffdominant} is better for everyone, 
they may fear that the other player deviates and hence, also choose 
the safer option. 

The predominant example of a social dilemma in the 
social sciences is the prisoner's dilemma. However, 
Camerer, for instance, argues that various situations that 
``are thought to be a prisoner's 
dilemma are actually a stag hunt game'' 
if the described situation ``evokes emotion, 
has enough synergy or excludability'' 
\parencite[376-377]{camerer_behavioral_2003}. Furthermore, infinitely
repeated prisoner's dilemma have attracted wide interest in the literature.
The famous book "The evolution of cooperation" of Robert Axelrod studies
what affects cooperation using computer simulations of repeated prisoner's
dilemmas \parencite{axelrod_evolution_2006}. Intringuingly, this version
also has a connection to the stag hunt game. \textcite{blonski_prisoners_2015}
show that in the repeated prisoner's dilemma risk-dominance and 
Pareto-dominance select different equilibria. Therefore, not only
the individual interest but also the uncertainty about the choice of the
opponent influences the outcome.
This connection stresses the importance of the equilibrium selection problem
represented by the stag hunt, 
as it underlies many important applications of game theory in all 
social sciences. 

The empirical evidence considered in section \ref{sec:experimentalevidence}
suggests that failure to coordinate on the best outcome for the group
in pure stag hunt games is common. 
In general, even preplay communication does not solve this.
However, situations in the real world that can be treated as a coordination
game may not involve the identical structure all the time. The
results of \textcite{rankin_strategic_2000} suggest that people can learn 
to coordinate when confronted with variation and hence,  

Evolutionary game theory does not describe all varities of the 
behavior in the laboratory. As a model should, it offers a simplifcation
that helps to understand some of the patterns observable when real people
interact. Additionally, it emphasizes that for a wide range
of behavioral rules coordination on the risk-dominant equilibrium
is more likely. While applied in various disciplines such as biology, 
psychology and political science, for example 
\textcite{friedman_economic_1998} argues that the full potential
of evolutionary game theory for economic application 
has not yet been unlocked. He suggests that "economists must re-adapt
evolutionary theory to economics" before it will be widely accepted 
\parencite[18]{friedman_economic_1998}. The underlying concept of behavior and
the simplification of an infinite population have to be carefully reflected
when applying the framework to economic problems. Additionally, questioning
the structure of the population is crucial. As seen, it may be pivotal
for coordination success. 
