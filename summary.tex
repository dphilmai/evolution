%%%%%%% 
%%
%% Summary and Outlook
%%
%%%%%%%


Stag hunt games do not just have to occur in the literal version Rousseau
described, even though there exists an interesting example provided by 
David A. Nolin and Michale S. Alvard on whale hunting in Lamalera, Indonesia
\parencite{nolin_meat-sharing_2000}.
Many situation in economics contain the coordination of actions
of individuals. Indicated in the introduction, usually the prisoner's dilemma is then used
as a framework to analyze why individuals may not be able to coordinate on the 
Pareto-dominant outcome. However, as Camerer
suggests, various situations that ``are thought to be a prisoner's dilemma 
are actually a stag hunt game'', if the described situation ``evokes emotion,
has enough synergy or excludability'' 
\parencite[376-377]{camerer_behavioral_2003}. Furthermore, repeated prisoner's
dilemma becomes a stag hunt game when only strategies grim trigger
and always defect are compared. \textcite{blonski_prisoners_2003} show that
this relationship is true for other strategies, considering the discrepancy
between Pareto-dominance and risk-dominance as equilibrium selection criteria 
for a wide range of parameter settings. 
Evolutionary game theory provides a theory in which the selection of an
equilibrium is endogenous once the dynamic is set into operation. Yet, it
does not offer an explanation for the initial conditions, which in fact 
determine on which equilibrium the population coordinates. Likewise, the
assumptions about the connection of individuals, discussed in 
\ref{sec:simplemodel}, the continuous population and the way agents
adjust their strategy need to be justified for the given application. 
