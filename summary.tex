%%%%%%% 
%%
%% Summary and Outlook
%%
%%%%%%%

Answering the question which factors affect social cooperation, one might
only be able to give a partial answer. 
This is beacuse the problem of cooperation can occur due to different
reasons.
The origin of the dilemma described in this thesis is the uncertainty 
of individuals about their counterparts choice. 
While one outcome guarantees everyone a higher 
payoff, they may fear that the other player deviates and hence, also choose 
the safer option. 
In the social sciences, the most frequently used framework for social dilemmas
involves a conflict between the individual interest and the interest of 
the group.
However, Camerer, for instance, argues that various situations that 
``are thought to be a prisoner's 
dilemma are actually a stag hunt game'' 
if the described situation ``is likely enough to be repeated, evokes emotion, 
has enough synergy or excludability'' 
\parencite[376-377]{camerer_behavioral_2003}. Hence, 
stag hunt game finds a wide range of different applications,
for example the coordination in  the macroeconomy \parencite{bryant_coordination_1994}, 
publishing decisions in academia \parencite{hanauske_evolutionare_2011} and
a discussion of the social contract \parencite{skyrms_stag_2004}.

The empirical evidence considered in section \ref{sec:experimentalevidence}
suggests that failure to coordinate on the payoff-dominant 
outcome for the group in pure stag hunt games is common. 
In general, even preplay communication does not solve this.
However, situations in the real world that can be treated as a coordination
game may not involve the identical structure all the time. The
results of \textcite{rankin_strategic_2000} suggest that people can learn 
to coordinate when confronted with variation and hence, we might not
have to be that pessimistic about coordination.
Evolutionary game theory does not describe all varities of the 
behavior in the laboratory. As a model should, it offers a simplifcation
that helps to understand some of the patterns observable when real people
interact. Additionally, it emphasizes that for a wide range
of behavioral rules coordination on the risk-dominant equilibrium
is more likely. While applied in various disciplines such as biology, 
psychology and political science, for example 
\textcite{friedman_economic_1998} argues that the full potential
of evolutionary game theory for economic application 
has not yet been unlocked. He suggests that "economists must re-adapt
evolutionary theory to economics" before it will be widely accepted 
\parencite[18]{friedman_economic_1998}. The underlying concept of behavior and
the simplification of an infinite population have to be carefully reflected
when applying the framework to economic problems. Furthermore, questioning
the structure of the population is important, as the underlying social
network may be pivotal for coordination success.
