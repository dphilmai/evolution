%%%%%%% 
%%
%% Summary and Outlook
%%
%%%%%%%


Stag hunt games do not just have to occur in the literal version Rousseau
described, even though there exists an interesting example provided by 
David A. Nolin and Michale S. Alvard on whale hunting in Lamalera, Indonesia
\parencite{nolin_meat-sharing_2000}.
Many situation in economics contain the coordination of actions
of individuals. Indicated in the introduction, usually the prisoner's dilemma 
is then used as a framework to analyze why individuals may not be able to 
coordinate on the Pareto-dominant outcome. However, as Camerer
suggests, various situations that ``are thought to be a prisoner's dilemma 
are actually a stag hunt game'', if the described situation ``evokes emotion,
has enough synergy or excludability'' 
\parencite[376-377]{camerer_behavioral_2003}. 
Furthermore, infinitely repeated prisoner's dilemma 
have attracted wide interest in the literature. 
For example, the popular book "The evolution of cooperation", 
Robert Axelrod studies what affects cooperation 
using computer simulations \textcite{axelrod_evolution_2006}. 
Intringuingly, there
exists a close connection to the stag hunt game. Allowing only 
for the strategy "grim trigger" and "defect only", the repeated prisoner's
dilemma turns into a stag hunt with a Pareto-dominant and a risk-dominant
equilibrium \parencite{blonski_prisoners_2015}. 
Recognizing the necessity to capture strategic risk, 
\textcite{blonski_prisoners_2015} advocate to use risk-dominance
to analyze repeated prisoner's dilemma. 
This connection stresses the importance of the equilibrium selection
problem, as it may underlie many important applications of game theory.

The framework provided by evolutionary game theory appealed to many different
scientific fields, such as biology, psychology, political science, and, 
of course economics. In economics, it was applicated to a wide range of 
topics. \textcite{hanauske_evolutionare_2011} used it to investigate the
socio-economic network of researches deciding whether to publish open-access
or in traditional journals.

In addition, evolutionary game theory provides a definite answer to
the equilibrium selection problem. The equilibrium selection is endogenous 
and only relies on the initial conditions of the population. 
Coordination on the Pareto-efficient outcome is less likely, as it 
has a smaller basin of attraction, i.e. fewer possible initial conditions
converge to it. Likewise, coordination failure seems to be prevalent in 
the laboratory. 
However, relaxing the assumptions of the approaches challenges this.
Confronted with similiar but slightly changed stag hunt games in 
\textcite{rankin_strategic_2000}, a convention to play the Payoff-dominant 
equilbrium emerged. One might argue that in reality people never play 
the exact same game over and over again, supporting the external validity
of this approach.
The usual approach of evolutionary game theory consists of agents that are 
fully connected to every other agent in their population. However,
the simple model in section \ref{sec:simplemodel} showed that 
even a naive network externality can change risk-dominance and the basins
of attraction of the equilibria. More rigorous treatments, using graphs to
model the structure of the population, may give further insight into the 
effect networks have.






