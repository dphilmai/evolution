\documentclass[12pt]{article}
\usepackage[utf8]{inputenc}
\usepackage[ngerman]{babel}
\usepackage[T1]{fontenc}
\usepackage{tikz}
\usepackage{pgfplots}
 % Useful for Physics related math
\usepackage{physics}
\usepackage{amsmath}
\usepackage{amssymb} 
\usepackage{amsthm}
        % packages that allow mathematical formatting
\usepackage{setspace}
\setlength\parindent{0pt}
        % set space and indent and indent
\usepackage{hyperref}
\hypersetup{
            colorlinks,
                linkcolor={red!50!black},
                    citecolor={blue!50!black},
                        urlcolor={blue!80!black}
                }
\usepackage[right=4cm,left=2.5cm,top=2.5cm,bottom=2.5cm]{geometry}

% New commands
\let\conju\overline
\newcommand\numberthis{\addtocounter{equation}{1}\tag{\theequation}}
\newcount\colveccount
\newcommand*\colvec[1]{
        \global\colveccount#1
        \begin{pmatrix}
                \colvecnext
        }
        \def\colvecnext#1{
                #1
                \global\advance\colveccount-1
                \ifnum\colveccount>0
                \\
                \expandafter\colvecnext
                \else
        \end{pmatrix}
        \fi
}
        % Columnvectors
\newtheorem{mydef}{Definition}

\pgfplotsset{compat=1.11}
\linespread{1.5}
\newcommand{\natnumb}{\mathbb{N}}
\newcommand{\realnumb}{\mathbb{R}}
\begin{document}
\section{The Stag hunt game}
As introduced, the stag hunt game is played by two players who choose their
strategies simultaneoulsy. Both have information about the strategies of the
other player and the payoffs they and their opponents receive. In game theory
such a game is called a \textit{normalform} game. Typically, such a game is
formalized as a Triplet $\Gamma = (I,\mathbb{S},F)$, with the set of players 
$I=\{1,2,...,n\}$, where $n$ is the total number of players, 
the pure strategy space of the game $S = \times_i S_i$
with the pure strategy space $S_i$ of an individual player 
$i \in I$ defined as $S_i = \{1,2,...,m_i\}$, where $m_i$ denotes the total
number of strategies available to player $i$, and the payoff function 
$F: S \rightarrow \realnumb^n$.
As this text focusses on the stag hunt game, which is a normalform game,
this definitons reduce to the following.
By setting $n=2$, we define the two hunters playing SH as $I=\{1,2\}$. A 
player $i \in I$  can choose from a set of pure strategies from his 
pure strategy space, $S_i$. Those strategies are called pure, because they 
differ from the later defined mixed strategies. In the SH game, 
each hunter can choose out of two pure strategies, namely, huntig the stag 
(strategy 1) or hunting hare (strategy 2).
Therefore, the pure strategy space is defined by $S_i = {1,2}$, which results
from definiton ???? by setting $m_i$. In the SH game both players choose from
the same set of stragies such that $S_1 =S_1=S$. Combining the strategy spaces
of the two players with the cartesian product
yields the definiton of the pure strategy space of the game
$\mathbb{S}= S \times S = S^2$.

the hunters do not just hunt for their pleasure \footnote{Even though the
situation would not differ, aslong as we assume the pleasure hunting a stag
to be higher than hunting hare, since economically payoffs are interpreted as
utility in the theory of households.}, but to seek 
a reward for their loot. This is captured by the definition of an individual
payoff function $F_i:S \rightarrow \realnumb$, which maps for a player $i \in
I$ to every state of the game $s=(s_1,s_2) \in S^2$, where $s_i$ is a 
pure strategy of player $i$, a payoff $F_i(s) \in \realnumb$. So
every outcome of the game, every combination of strategies the players could
individually choose, is defined. In the special case of two player, one 
can define a payoff matrix for player 1 $A \in \realnumb^{2 \times2}$ and for 
player 2  $B \in \realnumb^{2 \times2}$, where $A_{kl} = F_1(k,l)$ and $
B_{kl} = F_2(k,l)$ with the pure strategies $k,l \in S$. 











       
\end{document}

\

































