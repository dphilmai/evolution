\documentclass[12pt]{article}
\usepackage{todonotes}
\usepackage{tikz}
\usepackage{pgfplots}
 % Useful for Physics related math
\usepackage{physics}
\usepackage{amsmath}
\usepackage{amssymb} 
\usepackage{amsthm}
        % packages that allow mathematical formatting
\usepackage{setspace}
\setlength\parindent{0pt}
        % set space and indent and indent
\usepackage{hyperref}
\hypersetup{
            colorlinks,
                linkcolor={red!50!black},
                    citecolor={blue!50!black},
                        urlcolor={blue!80!black}
                }
\usepackage[a4paper,width=150mm,top=25mm,bottom=25mm]{geometry}

% New commands
\let\conju\overline
\newcommand\numberthis{\addtocounter{equation}{1}\tag{\theequation}}
\newcount\colveccount
\newcommand*\colvec[1]{
        \global\colveccount#1
        \begin{pmatrix}
                \colvecnext
        }
        \def\colvecnext#1{
                #1
                \global\advance\colveccount-1
                \ifnum\colveccount>0
                \\
                \expandafter\colvecnext
                \else
        \end{pmatrix}
        \fi
}
        % Columnvectors
\newtheorem{mydef}{Definition}

\pgfplotsset{compat=1.11}
\linespread{1.5}
\newcommand{\natnumb}{\mathbb{N}}
\newcommand{\realnumb}{\mathbb{R}}
\presetkeys{todonotes}{color=yellow,size=\scriptsize}{}
\begin{document}
\section{The Stag hunt game}
\todo{Motivation of traditional game theory}
As introduced, the stag hunt game is played by two players who choose their
strategies simultaneoulsy. Both have information about the strategies of the
other player and the payoffs they and their opponents receive. In game theory
such a game is called a \textit{normalform} game. Typically, such a game is
formalized as a Triplet $\Gamma = (I,\mathbb{S},F)$, with the set of players 
$I=\{1,2,...,n\}$, where $n$ is the total number of players, 
the pure strategy space of the game $S = \times_i S_i$
with the pure strategy space $S_i$ of an individual player 
$i \in I$ defined as $S_i = \{1,2,...,m_i\}$, where $m_i$ denotes the total
number of strategies available to player $i$, and the payoff function 
$F: S \rightarrow \realnumb^n$.
As this text focusses on the stag hunt game, which is a normalform game,
this definitons reduce to the following.
By setting $n=2$, we define the two hunters playing SH as $I=\{1,2\}$. A 
player $i \in I$  can choose from a set of pure strategies from his 
pure strategy space, $S_i$. Those strategies are called pure, because they 
differ from the later defined mixed strategies. In the SH game, 
each hunter can choose out of two pure strategies, namely, huntig the stag 
(strategy 1) or hunting hare (strategy 2).
Therefore, the pure strategy space is defined by $S_i = {1,2}$, which results
from definiton ???? by setting $m_i$. In the SH game both players choose from
the same set of stragies such that $S_1 =S_1=S$. Combining the strategy spaces
of the two players with the cartesian product
yields the definiton of the pure strategy space of the game
$\mathbb{S}= S \times S = S^2$.

In SH the hunters do not just hunt for their pleasure \todo{Hunter bewerten 
Ausgang des Spiels}\footnote{Even though the
situation would not differ, aslong as we assume the pleasure hunting a stag
to be higher than hunting hare, since economically payoffs are interpreted as
utility in the theory of households.}, but to seek 
a reward for their loot. This is captured by the definition of an individual
payoff function $F_i:S \rightarrow \realnumb$, which maps for a player $i \in
I$ to every state of the game $s=(s_1,s_2) \in S^2$, where $s_i$ is a 
pure strategy of player $i$, a payoff $F_i(s) \in \realnumb$. So
every outcome of the game, every combination of strategies the players could
individually choose, is defined. In the special case of two player, one 
can define a payoff matrix for player 1 $A \in \realnumb^{2 \times2}$ and for 
player 2  $B \in \realnumb^{2 \times2}$, where $A_{kl} = F_1(k,l)$ and $
B_{kl} = F_2(k,l)$ with the pure strategies $k,l \in S$. Using the 
representation above, one can indentify the payoff matrices for the player as
\begin{align}
        A = \mqty(a && b \\ c && d), \quad B = \mqty(a && c \\ b && d)
        \label{eq:matrix}
\end{align}

In the study of games, one is interested in defining subclasses of games.
For the game here in study the following definition is important: 
\begin{mydef}
        A two player game $\Gamma=(I,S_1 \times S_2, A,B)$ is called symmetric
        if the players of the game have the same strategy space $S_1=S_2=S$ and
        for the payoff matrices the condition $B=A^T$ holds. Therefore, the
        game is well-defined as $\Gamma=(I,S^2,A)$.
        \label{symmetry}
\end{mydef}
Clearly, the SH game is such a game. Both hunters have the same strategies 
available and the payoff matrices defined in \todo{Richtiges Nummerieren der 
Gleichungen} \eqref{eq:matrix} fulfill the required condition in Definition
\ref{symmetry}. Hence, it is irrelevant which player we label 1 or 2.


Usually a game is extended by the possibility of the players to play
\textit{mixed strategies}. Intuitively, in the analogy of SH, one can think of
such strategies as that a hunter decides whether to shoot the stag or the hare
by flipping a coin which shows head or tails, not necessarily with an equal
probabillity. Formally, every player of the game assigns a probabillity 
distribution over his pure strategies space.  
A strategy $x_i \in \Delta_i$ of player $i \in I$ 
is called a \textit{mixed strategy}, where $\Delta_i$ is the mixed strategy 
space 
\begin{align*}
        $\Delta_i = \{ x_i \in \realnumb^2 : \sum_{k \in S} x_{ik} = 1, x_{ik} \geq 0 \quad
\forall k \in S\}.
\end{align*}
\todo{Geometric Interpretation as a Simplex}
Since there is no restriction for any players choice of the probabillity 
distribution, in a symmetric game the mixed strategy spaces of the players
also equal. For the SH game this means $\Delta_1 = \Delta_2 = \Delta$.
With this notation a pure strategy can be interpreted as a mixed strategy
which assigns probabillity one to the pure strategy choosen and zero to all
other strategies. This is represented by the unit vectors of the simplex 
$\hat{e}_k \in \Delta$, $\hat{e}_1 = (1,0)^T$ is the pure strategy hunting stag 
and $\hat{e}_2 =(0,1)^T$ denots the pure strategy hunting hare.
Similiar to the pure strategy case, the mixed strategy spaces of the players 
$\Delta$ is combined to the mixed strategy space of the game $\Delta^2 =
\Delta \times \Delta$. From now on $x \in \Delta$ denotes a (mixed) strategy
chosen by player 1 and $y \in \Delta$ a (mixed) strategy of player 2.
Again similiar to the pure strategy case, the mixed strategy payoff 
$\hat{F}_i:\Delta^2 \rightarrow \realnumb$ maps to any state in the mixed strategy
space  $(x,y) \in \Delta^2$ a payoff $\hat{F}_i(x,y) \in \realnumb$.
With the matrix notation this is defined as: 
\begin{alignat*}{2}
        \hat{F}_1(x,y) &= x^T A y \\
        \hat{F}_2(x,y) &= x^T A^T y 
\end{alignat*}

Finally, the SH game is formally defined as a symmetric two-player normal form
game $\Gamma = (I=\{1,2\}, \Delta^2, \hat{F})$.

\subsection{Traditional concepts}
In describing the behavior of agents game theory developed a wide range of 
tools to solve this strategic interactions. \todo{Rationality assumption and
Equilibrium knowledge required for Nash equilibrium}

The \textit{best-reply} for player $i \in I$ to a strategy $y \in \Delta$ 
played by $j \neq i$ is defined as:
\begin{align}
        \beta(y) = \{x \in \Delta: \hat{F}(x,y) \geq \hat{F}(x',y), 
        \quad \forall x' \in \Delta\}
\end{align}
This formally assigns to each strategy of the other player the strategies
of player $i$ resulting in the highest payoff for player $i$. However, a player
must be capable of computing his best-reply to a given strategy.

The most famous and used traditional solution concept, the \textit{Nash 
equilibrium} assumes this kind of capability of each individual players. 
The Nash equilibrium was named by its proposer J. Nash in 1950. \todo{Citation
of Nash and history}

A Nash equilibrium for the SH game is defined as
\begin{mydef}
        A state of $(x^*,y^*) \in \Delta^2$ is called a Nash equilibrium if 
        it holds that\todo{Nicer form}
\begin{itemize}
        \item   $(x^*)^T A y^* \geq x^T A y^*, \quad \forall x \in \Delta$
        \item   $(x^*)^T A y^* \geq (x^*)^T A y, \quad \forall y \in \Delta$.
\end{itemize}
        It is called symmetric if $x^* = y^*$.
\end{mydef}
It is equivalent to say that both players play best-replies in a Nash 
equilibrium. One can proof that in every normal form games with mixed 
strategies a Nash equilibrium exists. This existence proof is due to Nash\todo{Citation}.

So, the SH game admits three symmetric Nash equilibria. The first one consists
of both players choosing strategy 1, hunting stag, $(\hat{e}_1,\hat{e}_1) \in
\Delta^2$. The second one is the state of 
both players choosing strategy 2, hunting hare, $(\hat{e}_2,\hat{e}_2)
\in \Delta^2$. Additionally to this Nash equilibria in pure strategies, there 
is a mixed strategy equilibrium. 




\subsection{Evolutionary concepts}













       
\end{document}


































