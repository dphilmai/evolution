\usepackage{tikz}
\usepackage{pgfplots}
 % Useful for Physics related math
\usepackage{physics}
\usepackage{amsmath}
\usepackage{amssymb} 
\usepackage{amsthm}
        % packages that allow mathematical formatting
\usepackage{setspace}
%\setlength\parindent{5pt}
        % set space and indent and indent
\usepackage{hyperref}
\hypersetup{
            colorlinks,
                linkcolor={black},
                    citecolor={black},
                        urlcolor={black}
                }
\usepackage[a4paper,left=40mm,top=20mm,bottom=20mm,right=20mm]{geometry}

% New commands
\let\conju\overline
\newcommand\numberthis{\addtocounter{equation}{1}\tag{\theequation}}
\newcount\colveccount
\newcommand*\colvec[1]{
        \global\colveccount#1
        \begin{pmatrix}
                \colvecnext
        }
        \def\colvecnext#1{
                #1
                \global\advance\colveccount-1
                \ifnum\colveccount>0
                \\
                \expandafter\colvecnext
                \else
        \end{pmatrix}
        \fi
}
        % Columnvectors
\newtheorem{mydef}{Definition}
\newenvironment{psmallmatrix}
        {\left(\begin{smallmatrix}}
        {\end{smallmatrix}\right)}

%%%%%%%%%%%%%%%%%%%%%%%%%%%%
%
% Possessivcite
%
%%%%%%%%%%%%%%%%%%%%%%%%%%
%\newcommand\posscite[1]{\citeauthor{#1}'s (\citeyear{#1})}
