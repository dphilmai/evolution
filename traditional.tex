\label{sec:traditional}
As introduced, the stag hunt game (SH) is played by two players who choose their
strategies without knowing each others choice.
Both have information about the strategies of the
other player and the payoffs they and their opponents receive. In game theory
such a game is called a \textit{normalform} game with complete information. 
Typically, such a game is formalized as a Triplet $\Gamma = (I,\mathbb{S},F)$, 
with the set of players $I=\{1,2,...,n\}$, where $n$ is the 
total number of players. The pure strategy space of the game 
$\mathbb{S} = \times_i S_i$ with the pure strategy space $S_i$ of an 
individual player  $i \in I$ defined as $S_i = \{1,2,...,m_i\}$, where $m_i$ 
denotes the total number of strategies available to player $i$. The payoff 
function of the game is denoted by $F: S \rightarrow \realnumb^n$.
As this text focuses upon the stag hunt game, these definitions reduce to the 
following.
By setting $n=2$, we define the set of two hunters playing SH as $I=\{1,2\}$. A 
player $i \in I$  can choose from a set of pure strategies from his 
pure strategy space, $S_i$. The strategies are called pure to distinguish
them between the later introduced mixed strategies. 
In the SH game, each hunter can choose one out of two pure strategies, namely, 
huntig the stag (strategy 1) or hunting hare (strategy 2).
Therefore, the pure strategy space is defined by $S_i = \{1,2\}$, which results
by setting $m_i=2$. In the SH game both players choose from
the same set of strategies such that $S_1 =S_2=S$. Combining the strategy spaces
of the two players with the cartesian product
yields the definition of the pure strategy space of the game
$\mathbb{S}= S \times S = S^2$. Games played by two players with two strategies
each, are often said to be 2x2 games.    
In the analogy of the story Rousseau told, the hunters do not just hunt
for their pleasure, but to sell their loot for a reward. 
This is captured by the definition of an individual payoff function for player
i, $F_i:S \rightarrow \realnumb$, which maps a payoff $F_i(s) \in \realnumb$ 
to every state of the game $s=(s_1,s_2) \in S^2$, where $s_i$ i s a pure
strategy of player $i$.\todo{this sentence}
The payoffs are usually interpreted as \textit{Von-Neumann} utility when 
played by individuals or households and are in terms of earnings regarding 
firms. 

In the special case of two player, one 
can define a payoff matrix for player 1, $A \in \realnumb^{2 \times2}$ and for 
player 2,  $B \in \realnumb^{2 \times2}$, where $A_{kl} = F_1(k,l)$ and $
B_{kl} = F_2(k,l)$ with the pure strategies $k,l \in S$. Using the 
representation in \ref{fig:sh}, one can identify the payoff matrices for 
the player as
\begin{align}
     A = \mqty(a && b \\ c && d), \quad B = \mqty(a && c \\ b && d)
        \label{eq:matrix}
\end{align}
For the game here in study, the following definition is important: 
\begin{mydef}
        A two player game $\Gamma=(I,S_1 \times S_2, A,B)$ is called symmetric
        if the players of the game have the same strategy space $S_1=S_2=S$ and
        for the payoff matrices the condition $B=A^T$ holds. Therefore, the
        game is well-defined as $\Gamma=(I,S^2,A)$.
        \label{symmetry}
\end{mydef}
Clearly, the SH game satisfies this condition. Both hunters have the same 
strategies available and the payoff matrices defined in 
\eqref{eq:matrix}  are the transpose of the other.
Hence, it is irrelevant which player is labeled as player 1 or 2, because 
they are identical with respect to their strategies and payoffs.
Usually a game is extended by the possibility of the players to play
\textit{mixed strategies}. 
The first intuition of a mixed strategy in the SH game would be hunters
choosing whether to shoot the stag or the hare by using a randomization
tool such as flipping a coin. However, randomization is not really
satisfying in the most applications \parencite{radner_private_1982}. An
alternative interpretation, outlined by Rubinstein, is that mixed 
strategies are actually deterministic, but seem to be random because they 
depend on not modelled private information of the individuals 
\parencite[914]{rubinstein_comments_1991}. In the evolutionary context,
discussed in section \ref{sec:evolutionarystaghunt}, mixed strategies have
a different, unproblematic interpretation. Besides the interpretation problem,
mixed strategies are appealing because they ensure the existence
of a Nash equilibrium, which is defined later in this text.
Formally, every player of the game assigns a probability distribution over
his pure strategies space. A strategy $x_i \in \Delta_i$ of 
player $i \in I$ is called a \textit{mixed strategy}, where $\Delta_i$ is 
the mixed strategy space 
\begin{align*}
        \Delta_i = \{ x_i \in \realnumb^2 : \sum_{k=1}^2 x_{ik} = 1, x_{ik} \geq 0 \quad
\forall k \in S\}.
\end{align*}
In a symmetric game the mixed strategy spaces of the players
equal, as both players are identical in assigning probabilities to their
pure strategies. For the SH game this means $\Delta_1 = \Delta_2 = \Delta$.
With this notation a pure strategy can be interpreted as a mixed strategy
which assigns probability one to the pure strategy chosen and zero to all
other strategies. This is represented by the unit vectors 
$e_k \in \Delta$, $e_1 = (1,0)^T$ is the pure strategy hunting stag 
and $e_2 =(0,1)^T$ denotes the pure strategy hunting hare.
The combined mixed strategy space of the game is $\Delta^2 = \Delta \times
\Delta$.
From now on $x \in \Delta$ denotes a (mixed) strategy
chosen by player 1 and $y \in \Delta$ a (mixed) strategy of player 2.
Again similar to the pure strategy case, the mixed strategy payoff 
$\hat{F}_i:\Delta^2 \rightarrow \realnumb$ maps to any state in the mixed strategy
space  $(x,y) \in \Delta^2$ a payoff $\hat{F}_i(x,y) \in \realnumb$.
With the matrix notation this is defined as: 
\begin{alignat*}{2}
        \hat{F}_1(x,y) &= x^T A y \\
        \hat{F}_2(x,y) &= x^T A^T y 
\end{alignat*}
Summarizing the notation, the SH game can be defined as a symmetric two-player
normal form game $\Gamma = (I=\{1,2\}, \Delta^2, \hat{F})$.

\subsection{Solution concepts}
\label{sec:traditionalconcepts}
In describing the behavior of agents, game theory developed a wide range of 
tools to solve this strategic interactions. First of all, for each individual
a best-reply is defined, simply describing the strategies available to him 
which yield the highest payoff against a given strategy of the other player.
Formally, the \textit{best-reply} for player $i \in I$ to a 
strategy $y \in \Delta$ played by $j \neq i$ is defined as:
\begin{align}
        \beta_i(y) = \{x \in \Delta: \hat{F}(x,y) \geq \hat{F}(x',y), 
        \quad \forall x' \in \Delta\}
\end{align}
Based on agents choosing a best-reply, the mostly used solution concept of
a normalform game is the \textit{Nash equilibrium}.
The Nash equilibrium was named by its proposer J. Nash in 1950 
and due to its prominence it is sometimes simply referred to as equilibrium.
In the following, the definition of a Nash equilibrium in a symmetric two
player game is stated. 
\begin{mydef}
        \label{def:nashequilibrium}
        A state of $(x^*,y^*) \in \Delta^2$ is called a Nash equilibrium if 
        it holds that
        \begin{alignat*}{2}
                (x^*)^T A y^* &\geq x^T A y^*, \quad \forall x \in \Delta \\
        \end{alignat*}
It is called symmetric if $x^* = y^*$. A Nash equilibrium is called a 
strict Nash equilibrium if the inequality is strict.
\end{mydef}
Only one equality is needed in a symmetric game as strategy space and
payoff function of each player are identical.
Equivalently to the definition, one can say that a strategy combination 
is a Nash equilibrium if all players choose a best-reply in that strategy 
combination and so not one player has the incentive to deviate from his 
choice, given the choice of the other players.
In \textcite{nash_equilibrium_1950}, Nash proofed the existence of such 
an equilibrium in any normal form game with mixed strategies. 
The SH game admits three symmetric Nash equilibria. The first one consists
of both players choosing strategy 1, hunting stag, $(e_1,e_1) \in
\Delta^2$, where both players receive the payoff $a$. 
None of both has an incentive to change his decision. For both
playing the pure strategy 1 is a best-reply because a unilateral deviation 
leads to the payoff $c \leq a$.
The second equilibrium is the state of 
both players choosing strategy 2, hunting hare, $(e_2,e_2)
\in \Delta^2$. Again, by definition, both play a best-reply and a unilateral 
deviation of a player results in a lower payoff $b<d$.
Additionally to this Nash equilibria in pure strategies, there is a symmetric
mixed-strategy equilbrium where both players assign a probability 
$q^*=(\alpha_2,\alpha_1 + \alpha_2)$ to strategy 1. 
To see this, a player is indifferent between choosing one of his pure
strategies against $y$, $\hat{F}(e_1,y) = \hat{F}(e_2,y)$, exactly
when $y=(q^*,1-q^*)^T$.

For further analysis it is convenient to use the fact that Nash equilibria 
are only defined by a difference inequality. 
Hence, any affine transformation of the payoff matrix does not change the 
best-replies of the players and so does not effect the set Nash equilibria 
of the game \parencite[17-19]{weibull_evolutionary_1997}. 
Furthermore, adding a constant to every entry in a column of the payoff matrix, 
called a \textit{local shift}, does also not effect the payoff differences
a player considers when valuating which strategy is better against a given
strategy of the other player.
Formally, let $A_{ij}$ denote the elements of the player's payoff matrix. 
A local shift to column $j^*$ transform this payoff matrix into the payoff 
matrix $A^*$:
\begin{align*}
        A^*_{ij} =
        \begin{cases}
                A_{ij} + v & \text{for}\ j=j^*, v \in \realnumb \\
                A_{ij}
        \end{cases}
\end{align*}
Applying this to the stag hunt game one can use two local shifts to turn 
the payoff matrix into a diagonal matrix with the elements $A_{11}=\alpha_1$ 
and $A_{22}=\alpha_2$, where $\alpha_1=a-c$ and $\alpha_2=d-b$. 
\textcite[28]{weibull_evolutionary_1997} groups all symmetric 2x2 games into 
four categories with different equilibrium properties. 
The stag hunt game is considered as a coordination 
game, with $\alpha_1, \alpha_2 > 0$ and in contrast to the class of games
with dominated strategies, for example the PD, the solution of the game
is not so obvious as there are multiple Nash equilibria.
That the Nash equilibrium does not point to a unique solution for the stag
hunt game will be discussed in section \ref{sec:equilibriumselection}.

\subsection{Evolutionary concepts}
The stag hunt game has multiple Nash equilibria. Asking which equilibrium
will be played, seems to be reasonable. Answering this question, game 
theorists tried to construct refinements of the Nash equilibrium in cases where 
some equilibria seemed to be unconvincing. Motivated by the context of 
evolution in animal populations \textcite{smith_lhe_1973} constructed the 
refinement of a \textit{Evolutionary stable strategy} (ESS).  
In the biological context, a game is played 
by a large population randomly matched against each other.
An ESS is a strategy that, once adopted by the whole population, cannot
be invaded by any other strategy. 
This definition of an ESS makes clear that it is actually a refinement of the 
Nash equilibrium.
\begin{mydef}
        A strategy $x \in \Delta$ is called a evolutionary stable strategy 
        (ESS) if it holds that
        \begin{alignat}{2}
                \label{eq:essstrict}
                \hat{F}(x,x) &\geq \hat{F}(y,x) \forall y \\ 
                \hat{F}(y,x) &= \hat{F}(x,x) \Rightarrow  
                \hat{F}(y,y) < \hat{F}(x,y) \forall y \neq x \label{eq:essstrict2}
        \end{alignat}
\end{mydef}
Equation \eqref{eq:essstrict}
requires an ESS $x$ to be a better reply to itself than any other strategy $y$.
If there exists another strategy $y$ with the same payoff, equation 
\eqref{eq:essstrict2} demands that it is better to choose $x$ against $y$ than
choosing $y$ against itself.
Indeed, seen in \eqref{eq:essstrict}, an ESS is used in a Nash equilibrium as 
it needs to be a best-reply to satisfy the ESS condition. In addition, 
a strategy used in a strict Nash equilibrium is always an ESS, 
since it directly satisfies 
\eqref{eq:essstrict}. 
In the stag hunt game, not all strategies are evolutionary stable.
The pure strategies $e_1$ and $e_2$ are evolutionary stable,
since both are strict symmetric Nash equilibria and so 
satisfy condition \eqref{eq:essstrict}. 
The mixed strategy Nash equilibrium is not an ESS.  
Let again $y=(q^*,1-q^*)^T$, with $q=\frac{\alpha_2}{\alpha_1+\alpha_2}$.
Considering for example the pure strategy $e_1$, 
equation \ref{eq:essstrict} is $\hat{F}(y,e_1)= \hat{F}(y,y)$.  
But as equation \eqref{eq:essstrict2} is violated,
$\hat{F}(y,e_1) = \frac{\alpha_1 \alpha_2}{\alpha_1+\alpha_2}
< \alpha_1 = \hat{F}(e_1,e_1)$, the Nash equilibrium with mixed strategies 
is not an ESS.

In section \ref{sec:replicatordynamic}, the usefulness of the ESS as solution
concept for the dynamic framework of evolutionary game theory will come 
into sharper relief.

\subsection{Equilibrium Selection}
\label{sec:equilibriumselection}
Clearly, game theory has the aspiration to provide a unique solution in every
strategic interaction.
As seen in the stag hunt game, the mostly used solution concept, 
the Nash equilibrium, and refinements such as the ESS are
not sufficient to select a unique equilibrium, even in the case of a simple
2x2 SH game. This does not satisfy game theorists, since it is not clear which
equilibrium is finally played by the agents. Or how 
Weibull puts it, this kind of coordination games 
``caused\footnote{And still causes.} game theorists and users of 
noncooperative game theory a fair amount of frustration'' 
\parencite[30]{weibull_evolutionary_1997}. 

A closer look at the two pure Nash equilibria in the Stag hunt game 
shows their difference. Considering the normal form representation in figure 
\ref{fig:sh}, the Nash equilibrium, in which both players choose strategy one 
has the highest payoff
for both players. $(e_1,e_2)$ is then said to 
\textit{payoff-dominate} or to \textit{Pareto-dominate} 
$(e_2,e_2)$. Equivalently, the equilibria are said to be
\textit{Pareto-ranked}.
In the Prisoners Dilemma, briefly described in the introduction, 
the payoff-dominant outcome is not a 
Nash equilibrium, due to the fact that every agent has the incentive to 
deviate and accept the deal of the authority.
Contrary to that, in SH, every agent plays a best-reply 
in the Pareto efficient outcome, but since there is another Nash equilibrium 
it is not clear which one is played based on equilibrium play alone. 
Following \textcite[57]{schelling_strategy_1960}, one may argue that 
Pareto-dominance characterizes $(e_1,e_1)$ as a focal point, 
an outcome of a game that is psychologically prominent, of 
the game, helping the players to coordinate on this outcome.
However, the equilibrium point $(e_2,e_2)$ exhibits the feature 
of \textit{Risk-Dominance}. 
\textcite{harsanyi_general_1988} defined this selection criterion 
based on the
risk for the players associated with an equilibrium point. Formally, in the
SH game, 
\begin{mydef}
The Nash equilibrium $x=(e_2,e_2)$ \textit{risk-dominates} 
the Nash equilibrium $y=(e_1,e_1)$, if $(d-b)^2 > (a-c)^2$.
         \label{eq:riskdom}
 \end{mydef}
The squares on the left and the right side of the inequality are named
\textit{Nash products}.
As mentioned in \textcite{weibull_evolutionary_1997} a NE risk-dominates, 
if it is Pareto-efficient  in the reduced payoff version of the game, 
associated with the diagonal
payoff matrix, as the definition \eqref{eq:riskdom} 
corresponds to $\alpha_1 < \alpha_2$.
So both pure Nash equilibria have a certain appeal for game theorists to be
favored as an outcome of the game. The question is, do people rather play
safe and coordinate on the hare hunting equilibrium, terminating in a social
dilemma as there exists an outcome in which everyone of them is better off.
Or are they able to recognize the Payoff-dominance of stag hunting, 
trusting each other not to react to a hare passing by.
Indeed, there is no consensus which equilibrium will be played. 
Although Harsanyi and Selten introduced the Risk-dominance criterion in 
\textcite{harsanyi_general_1988}, their general theory of equilibrium 
selection favors Payoff-dominance in the Stag hunt case. They argue that
``if each player knows the other to be fully rational [...] they should trust
each other to be fully rational'' \parencite[89]{harsanyi_general_1988}.
Further on, Harsanyi and Selten expect players to coordinate on the
Pareto-dominant equilibrium if they are allowed to communicate beforehand,
sometimes referred to as \textit{cheap talk} when it does not directly affect
the payoffs \parencite[104]{farrell_cheap_1996}, 
because ``an agreement to do so is self-stabilizing''. 
In contrast to that,
\textcite{aumann_nash_1990} points out, that a message from player 1 to 
player 2, saying that he intends to coordinate on the Payoff-dominant 
equilibrium, does not in general contain useful information for player 2. 
Indeed, when $c$ is strictly greater than $d$, a player wants the opposite 
player to choose strategy 1, independently of his choice.
Despite this argument, 
\textcite[114]{farrell_cheap_1996} expect that 
``cheap talk will do a good deal to
bring Artemis and Calliope\footnote{The names
\textcite{farrell_cheap_1996} used for player 1 and player 2.} 
to the stag hunt''. The experimental evidence will give a hint how cheap
talk affects real players. 
``Convinced by Aumann's argument'', Harsanyi formulated 1995 a new theory
of equilibrium selection, solely focussing on risk-dominance as selection
criterion, hence favorising the hare equilibrium in SH game 
\parencite[92,94,96]{harsanyi_new_1995}. 

Even though the evolutionary refinement ESS, simply coinciding with
the pure equilibria, does not help as a selection
criterion in this case, the dynamic evolutionary framework, specified in the
next section, gives an answer to the question which equilibrium is chosen. 

